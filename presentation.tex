\documentclass{beamer}

\usetheme{Boadilla}

\usepackage{listings}

\usepackage{hyperref}
\hypersetup{colorlinks=true}
\title{C++14 is coming}
%\subtitle{}
\author{Kurbakov Dmytro}

\begin{document}
\begin{frame}
\titlepage
\end{frame}

\begin{frame}{Overview}
\tableofcontents
\end{frame}


\section{Removed functions}

\begin{frame}
\begin{center}
\huge Removed functions
\end{center}
\end{frame}

\begin{frame}[fragile]{std::gets}
Reads stdin into given character string until a newline character is found or end-of-file occurs.
\begin{block}{C++11}
\begin{lstlisting}[firstnumber=1, label=glabels, xleftmargin=5pt, language=C++] 
char* std::gets( char* str );
\end{lstlisting}
\end{block}
This unsafe I/O function from the C library is no longer available.
\vfill
Source: \href{http://www.open-std.org/jtc1/sc22/wg21/docs/papers/2013/n3733.pdf}{ISO/IEC CD 14882, C++ 2014, National Body Comments }
\end{frame}

\begin{frame}[fragile]{std::rand}
Returns a pseudo-random integral value between ​0​ and RAND\_MAX (0 and RAND\_MAX included).
\begin{block}{C++11}
\begin{lstlisting}[firstnumber=1, label=glabels, xleftmargin=5pt, language=C++] 
// use current time as seed for random generator
std::srand(std::time(nullptr));
std::cout << std::rand() << '\n';
\end{lstlisting}
\end{block}
This low-quality random number facility from the C library is discouraged in favour of the random library.
\vfill
Source: \href{http://www.open-std.org/jtc1/sc22/wg21/docs/papers/2014/n3924.pdf}{Discouraging rand() in C++14, v2}
\end{frame}

\section{Language extentions}


\begin{frame}
\begin{center}
\huge Language extentions
\end{center}
\end{frame}


\begin{frame}{Generalized return type deduction}
C++11 \textbf{\textit{auto}} was introduced with limited area of usage. 

C++11 missed \textbf{\textit{auto}} for return types... 
\begin{quote}
...due to time constraints, as the drafting didn't address various questions and concerns that the Core WG had.
\end{quote}

C++14 \textbf{\textit{auto}} can be used as return type... but not for virtual calls
\begin{quote}
It would be possible to allow return type deduction for virtual functions, but that would complicate both override checking and vtable layout, so it seems preferable to prohibit this. 
\end{quote}

\vfill
Source: \href{http://www.open-std.org/jtc1/sc22/wg21/docs/papers/2013/n3638.html}{Return type deduction for normal functions}
\end{frame}

\begin{frame}[fragile]{Generalized return type deduction}
\begin{block}{C++11}
\begin{lstlisting}[firstnumber=1, label=glabels, xleftmargin=5pt, language=C++] 
auto f() -> int { return foo() * 42; }
\end{lstlisting}
\end{block}

\begin{block}{C++14}
\begin{lstlisting}[firstnumber=1, label=glabels, xleftmargin=5pt, language=C++] 
auto f() { return foo() * 42; }
auto g() {
    if( expr ) { return foo() * 42; }
    // multiple returns (types must be the same)
    return bar.baz(84); 
}
\end{lstlisting}
\end{block}
\end{frame}

\begin{frame}{decltype(auto)}
\textbf{\textit{decltype(auto)}} is primarily useful for deducing the return type of forwarding functions and similar wrappers and not intended to be a widely used feature beyond that.

\vfill
Source: \href{http://www.open-std.org/jtc1/sc22/wg21/docs/papers/2013/n3638.html}{Return type deduction for normal functions}
\end{frame}

\begin{frame}[fragile]{decltype(auto)}
\begin{block}{Given}
\begin{lstlisting}[firstnumber=1, label=glabels, xleftmargin=5pt, language=C++] 
string  lookup1();
string& lookup2();
\end{lstlisting}
\end{block}

\begin{block}{C++11}
\begin{lstlisting}[firstnumber=1, label=glabels, xleftmargin=5pt, language=C++] 
string  wrapper_1() {return lookup1();}
string& wrapper_2() {return lookup2();}
\end{lstlisting}
\end{block}

\begin{block}{C++14}
\begin{lstlisting}[firstnumber=1, label=glabels, xleftmargin=5pt, language=C++] 
decltype(auto) wrapper_1() {return lookup1();}
decltype(auto) wrapper_2() {return lookup2();}
\end{lstlisting}
\end{block}
\end{frame}

\begin{frame}[fragile]{decltype(auto)}
Important: \textbf{\textit{decltype(auto)}} is sensitive to how you write the return statement

\begin{block}{Quiz: what return these two functions?}
\begin{lstlisting}[firstnumber=1, label=glabels, xleftmargin=5pt, language=C++] 
decltype(auto) foo() {
    auto str = lookup1(); return str;
}
decltype(auto) boo() {
    auto str = lookup1(); return(str);
}
\end{lstlisting}
\end{block}
\vfill 
\href{https://isocpp.org/wiki/faq/cpp14-language#decltype-auto}{Answer}
\end{frame}

\begin{frame}{decltype(auto) vs. auto}
When should we use \textbf{auto} and when \textbf{decltype(auto)}?

\begin{block}{Rules of thumb}
\begin{itemize}
\item Use \textbf{auto} if a reference type would never be correct.
\item Use \textbf{decltype(auto)} only if a reference type could be correct.
\end{itemize}
\end{block}

\vfill
Source: \href{https://www.aristeia.com/TalkNotes/C++TypeDeductionandWhyYouCareCppCon2014.pdf}{C++ Type deduction and why you care. S. Mayers}
\end{frame}

\begin{frame}{Generalized lambda captures}
\begin{itemize}
\item C++11 no support fo capturing by move. 
\item C++14 generalized lambda capture (capture by move, define arbitrary new local variables in the lambda object).
\end{itemize}

\vfill
Source: \href{http://www.open-std.org/jtc1/sc22/wg21/docs/papers/2013/n3648.html}{Wording Changes for Generalized Lambda-capture}

\end{frame}


\begin{frame}[fragile]{Generalized lambda captures}
\begin{block}{C++14: capture by move}
\begin{lstlisting}[firstnumber=1, label=glabels, xleftmargin=5pt, language=C++]
// a unique_ptr is move-only 
auto u = make_unique<Type>(arg1, arg2);
// move the unique_ptr into the lambda 
go.run( [ u=move(u) ] { foo( u ); } );
\end{lstlisting} 
\end{block}

\begin{block}{C++14: define new variables}
\begin{lstlisting}[firstnumber=1, label=glabels, xleftmargin=5pt, language=C++] 
int x = 4;
int z = [&r = x, y = x+1] {
r += 2; // set x to 6; "R is for Renamed Ref"
return y+2;  // return 7 to initialize z
}(); // invoke lambda
\end{lstlisting}
\end{block}
\end{frame}

\begin{frame}{Generic lambdas}
\begin{itemize}
\item Lambda function parameters can now be auto to let the compiler deduce the type.
\item Fix UB. See the example from the Appendix A.
\end{itemize}

\vfill
Source: \href{http://www.open-std.org/jtc1/sc22/wg21/docs/papers/2013/n3559.pdf}{Proposal for Generic (Polymorphic) Lambda Expressions}
\end{frame}

\begin{frame}[fragile]{Generic lambdas}
\begin{block}{C++14}
\begin{lstlisting}[firstnumber=1, label=glabels, xleftmargin=5pt, language=C++] 
for_each(begin(v), end(v), [](const auto& x) { 
	cout << x; 
} );
sort( begin(w), end(w), 
	[](const auto& a, const auto& b) { 
	return *a<*b; 
} );
auto size = [](const auto& m) { 
	return m.size(); 
};
\end{lstlisting}
\end{block}
\end{frame}

\begin{frame}[fragile]{Default arguments for lambdas}
Lambda expressions can now have default arguments
\begin{lstlisting}[firstnumber=1, label=glabels, xleftmargin=5pt, language=C++] 
auto f = [](int a = 10){}; 
f();
\end{lstlisting}
\vfill
Source: \href{http://www.open-std.org/jtc1/sc22/wg21/docs/cwg_defects.html#974}{Default arguments for lambdas}
\end{frame}


\begin{frame}{Variable templates}
Idea: Simplify definitions and uses of parameterized constants, allow the definition and uses of constexpr variable templates

\vfill
Source: \href{http://www.open-std.org/jtc1/sc22/wg21/docs/papers/2013/n3651.pdf}{Variable Templates (Revision 1)}
\end{frame}

\begin{frame}[fragile]{Variable templates}
\begin{block}{C++14}
\begin{lstlisting}[firstnumber=1, label=glabels, xleftmargin=5pt, language=C++] 
// variable template
template<class T>
constexpr T pi = T(3.1415926535897932385L);
// function template
template<class T>
T circular_area(T r)
{return pi<T> * r * r;}
\end{lstlisting}
\end{block}
\end{frame}

\begin{frame}{Extended constexpr}
\begin{itemize}
\item Allow declarations within constexpr functions, other than
\begin{itemize}
\item static or thread\_local variables 
\item uninitialized variables
\end{itemize}
\item Allow if and switch statements (but not goto)
\item Allow all looping statements: for (including range-based for), while, and do-while
\item Allow mutation of objects whose lifetime began within the constant expression evaluation.
\item Remove the rule that a constexpr non-static member function is implicitly const.
\end{itemize}

\vfill
Source: \href{http://www.open-std.org/jtc1/sc22/wg21/docs/papers/2013/n3652.html}{Relaxing constraints on constexpr functions constexpr member functions and implicit const}
\end{frame}

\begin{frame}[fragile]{Extended constexpr}
\begin{block}{C++14}
\begin{lstlisting}[firstnumber=1, label=glabels, xleftmargin=5pt, language=C++] 
constexpr 
int my_strcmp(const char* str1, 
              const char* str2 ) {
  int i = 0;
  for(;str1[i] 
       && str2[i] 
       && str1[i] == str2[i]; ++i)
      { }
  if( str1[i] == str2[i] ) return 0;
  if( str1[i] < str2[i] ) return -1;
  return 1;
}
\end{lstlisting}
\end{block}
\end{frame}

\begin{frame}{The [[deprecated]] attribute}
\begin{itemize}

\item The attribute-token deprecated can be used to mark names and entities whose use is still allowed, but is discouraged for some reason.

\item The attribute may be applied to the declaration of a class, a typedef-name, a variable, a non-static data member, a function, an enumeration, or a template specialization.
\end{itemize}

\vfill
Source: \href{http://www.open-std.org/jtc1/sc22/wg21/docs/papers/2013/n3760.html}{[[deprecated]] attribute}

\end{frame}

\begin{frame}[fragile]{The [[deprecated]] attribute}
\begin{block}{C++14}
\begin{lstlisting}[firstnumber=1, label=glabels, xleftmargin=5pt, language=C++] 
struct [[deprecated]] S;
class Foo() {
[[deprecated]] void f();
};
using PS [[deprecated]] = S*;
union U { [[deprecated]] int n; };
namespace [[deprecated]] NS { int x; }
\end{lstlisting}
\end{block}
\end{frame}

\begin{frame}[fragile]{Tuple access by type}
Allowing tuples to be addressed by type as well as by numerical index.

\begin{lstlisting}[firstnumber=1, label=glabels, xleftmargin=5pt, language=C++] 
tuple<char, char, int> t('a', 'b', 1);
int i = get<int>(t); // i == 1
int j = get<2>(t); // j == 1
char s = get<char>(t); // Compile-time error
\end{lstlisting}

\vfill
Souce: \href{http://www.open-std.org/jtc1/sc22/wg21/docs/papers/2013/n3670.html}{Wording for Addressing Tuples by Type: Revision 2} 
\end{frame}

\begin{frame}[fragile]{Constexpr member functions without const}
A constexpr member function is no longer implicitly const. Only for data members does constexpr imply const now.	

\begin{block}{C++11}
\begin{lstlisting}[firstnumber=1, label=glabels, xleftmargin=5pt, language=C++] 
// const function
constexpr size_t size() noexcept;
\end{lstlisting}
\end{block}

\begin{block}{C++14}
\begin{lstlisting}[firstnumber=1, label=glabels, xleftmargin=5pt, language=C++] 
// non-const function
constexpr size_t size() noexcept;
// const function
constexpr size_t size() const noexcept;
\end{lstlisting}
\end{block}
\vfill
Source: \href{http://www.open-std.org/jtc1/sc22/wg21/docs/papers/2013/n3669.pdf}{Fixing constexpr member functions without const}
\end{frame}

	
\section{Library extentions}


\begin{frame}
\begin{center}
\huge Library extentions
\end{center}
\end{frame}

\begin{frame}{Shared locking}
The class shared\_lock is a general-purpose shared mutex ownership wrapper allowing deferred locking, timed locking and transfer of lock ownership. Locking a shared\_lock locks the associated shared mutex in shared mode.

As the result:
\begin{itemize}
\item Seven constructors to unique\_lock$<$Mutex$>$ were added
\item A new header shared\_mutex was added
\begin{itemize}
\item class shared\_mutex;
\item class upgrade\_mutex;
\item template $<$class Mutex$>$ class shared\_lock;
\item template $<$class Mutex$>$ class upgrade\_lock;
\end{itemize}
\end{itemize}

\vfill
Source: \href{http://www.open-std.org/JTC1/SC22/WG21/docs/papers/2013/n3659.html}{Shared locking in C++}
\end{frame}

\begin{frame}[fragile]{make\_unique}
No comments =)

\begin{block}{C++14}
\begin{lstlisting}[firstnumber=1, label=glabels, xleftmargin=5pt, language=C++] 
auto p1 = make_shared<widget>();
auto p2 = make_unique<widget>();
\end{lstlisting}
\end{block}

\vfill
Source: \href{http://www.open-std.org/jtc1/sc22/wg21/docs/papers/2013/n3656.htm}{JTC1/SC22/WG21 N3656}
\end{frame}

\begin{frame}{Constant, reverse and constant reverse iterators}
More iterators in C++14

\begin{itemize}
\item Constant iterators: std::cbegin, std::cend
\item Reverse iterators: std::rbegin, std::rend
\item Constant reverse iterators: std::rcbegin, std::rcend
\end{itemize}
\end{frame}

\begin{frame}{Overload for std::equal, std::missmatch, std::is\_permutation}
Algorithms that operate on two ranges get new overloads that take begin and end iterators for both ranges rather than requiring the second range to be sufficiently long.
\vfill
Source: \href{http://www.open-std.org/jtc1/sc22/wg21/docs/papers/2013/n3671.html}{Making non-modifying sequence operations more robust: Revision 2}
\end{frame}

\begin{frame}[fragile]{std::exchange}

Replaces the value of obj with new\_value and returns the old value of obj.

\begin{lstlisting}[firstnumber=1, label=glabels, xleftmargin=5pt, language=C++] 
template<class T, class U = T>
T exchange( T& obj, U&& new_value );
\end{lstlisting}
Important: T must meet the requirements of MoveConstructible. Also, it must be possible to move-assign objects of type U to objects of type T
\vfill
Source: \href{http://www.open-std.org/jtc1/sc22/wg21/docs/papers/2013/n3668.html}{exchange utility function, revision 3}
\end{frame}

\begin{frame}{Compile-time integer sequences}
The class template std::integer\_sequence represents a compile-time sequence of integers.
Consis of:
\begin{itemize}
\item std::index\_sequence
\item std::integer\_sequence
\item std::make\_index\_sequence
\item std::make\_integer\_sequence
\end{itemize}
\vfill
Source: \href{http://www.open-std.org/jtc1/sc22/wg21/docs/papers/2013/n3658.html}{Compile-time integer sequences}
\end{frame}


\section{C++14 and NavKit. Open discussion}


\begin{frame}{C++14 and NavKit. Open discussion}
\begin{columns}[T]
\begin{column}{0.5\textwidth}
Language:
\begin{itemize}
\item Generalized return type deduction
\item decltype(auto)
\item Generalized lambda captures
\item Generic lambdas 
\item Default arguments for lambdas
\item Variable templates 
\item Extended constexpr
\item The [[deprecated]] attribute
\item Tuple access by type
\item Constexpr member functions without const
\end{itemize}
\end{column}
\begin{column}{0.5\textwidth}
Library
\begin{itemize}
\item Shared locking
\item std::make\_unique
\item const and reverse iterators
\item Overload for std::equal, std::missmatch, std::is\_permutation
\item std::exchange
\item Compile-time integer sequences
\end{itemize}
\end{column}
\end{columns}
\end{frame}	

\begin{frame}{Whant to know more?}

Some useful links for C++14
\vfill
\begin{itemize}
\item \href{https://isocpp.org/files/papers/p1319r0.html}{Changes between C++11 and C++14}
\item \href{https://isocpp.org/wiki/faq/cpp14-language}{C++14 Language Extensions}
\item \href{https://isocpp.org/wiki/faq/cpp14-library}{C++14 Library Extensions}
\item \href{https://www.oreilly.com/library/view/effective-modern-c/9781491908419/ch01.html}{Effective Modern C++ by Scott Meyers, Chapter 1. Deducing Types}
\end{itemize}
\end{frame}

\end{document}